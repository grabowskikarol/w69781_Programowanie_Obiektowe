% ********** Rozdział 1 **********
\chapter{Opis założeń projektu}
\section{Cele projetu}
%\subsection{Tytuł pierwszego podpunktu}

Celem niniejszego projektu jest opracowanie zaawansowanego systemu informatycznego, którego głównym zadaniem będzie wsparcie procesów związanych z zarządzaniem ofertami turystycznymi, rezerwacjami oraz rozliczeniami w biurze podróży. Projekt zakłada stworzenie elastycznego narzędzia umożliwiającego bieżącą modyfikację i aktualizację oferty turystycznej poprzez funkcjonalności pozwalające na tworzenie, edytowanie oraz usuwanie poszczególnych pozycji oferty, co umożliwi dynamiczne dostosowywanie się do zmieniających się warunków rynkowych.
System zostanie wyposażony w moduł rezerwacyjny, który umożliwi klientom samodzielne dokonywanie rezerwacji biletów, a jednocześnie pozwoli pracownikom biura na efektywne zarządzanie procesem rezerwacyjnym. Wdrożenie funkcjonalności rozliczeniowych stanowi kolejny kluczowy element projektu, który ma na celu usprawnienie kontroli płatności oraz rozliczeń finansowych, co przyczyni się do zwiększenia przejrzystości oraz efektywności zarządzania finansami w biurze podróży.


\section{Wymagania funkcjonale i niefunkcjonalne}

\noindent \textbf{Wymagania funkcjonalne}
\begin{itemize}
    \item Zarządzanie ofertami: Użytkownicy systemu (pracownicy biura) mogą dodawać nowe oferty, edytować ich parametry oraz usuwać je z bazy.
    \item Prezentacja oferty: Klienci mają możliwość przeglądania pełnych informacji o dostępnych ofertach, takich jak nazwa, destynacja, cena i liczba dostępnych miejsc.
    \item Rezerwacje: System umożliwia klientom dokonywanie rezerwacji wybranych ofert, zapisując te dane w wewnętrznej bazie.
    \item Modyfikacja rezerwacji: Pracownicy biura mogą wprowadzać zmiany w rezerwacjach lub je anulować.
    \item Obsługa płatności: System rejestruje płatności, generuje proste faktury oraz umożliwia przegląd historii transakcji.
    \item Zarządzanie użytkownikami: Możliwość tworzenia kont, przypisywania im określonych ról oraz ograniczania dostępu do niektórych funkcji systemu.
\end{itemize}

\noindent \textbf{Wymagania niefunkcjonalne }
\begin{itemize}
    \item Wydajność: System powinien działać sprawnie nawet przy dużej liczbie ofert i rezerwacji. 
    \item Skalowalność: Aplikacja musi umożliwiać łatwe dodawanie nowych modułów i funkcjonalności.
    \item Bezpieczeństwo: Dane osobowe klientów i informacje finansowe muszą być odpowiednio chronione, co w przyszłości można rozszerzyć o mechanizmy szyfrowania i autoryzacji.
    \item Przyjazny interfejs: Nawet jeśli obecnie jest to interfejs konsolowy, system powinien być intuicyjny i łatwy w obsłudze.
    \item Kompatybilność: System oparty na technologii .NET musi działać na różnych systemach operacyjnych oraz być zgodny z aktualnymi standardami.
    \item Niezawodność: System powinien minimalizować ryzyko awarii oraz umożliwiać tworzenie kopii zapasowych danych.
\end{itemize}
 


% ********** Koniec rozdziału **********
