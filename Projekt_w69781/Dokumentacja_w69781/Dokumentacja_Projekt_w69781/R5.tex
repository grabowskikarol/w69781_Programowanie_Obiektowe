% ********** Rozdział 4 **********
\chapter{Podsumowanie}

Projekt systemu zarządzania biurem podróży, napisany w języku C\#, stanowi przykład rozwiązania informatycznego opartego na modularnej architekturze, w której kluczową rolę odgrywają trzy główne warstwy: modele danych, logika biznesowa oraz interfejs użytkownika. Modele danych, takie jak Oferta, Rezerwacja, Płatność oraz Użytkownik, służą do przechowywania podstawowych informacji niezbędnych dla funkcjonowania systemu. Dedykowane klasy zarządzające, odpowiadające za operacje CRUD oraz dodatkowe funkcjonalności, jak tworzenie kopii zapasowych, generowanie faktur czy kontrola dostępu, umożliwiają sprawną obsługę danych. Warstwa interfejsu, mimo że opiera się na konsolowym menu, zapewnia intuicyjną i przejrzystą interakcję, umożliwiając użytkownikowi łatwą nawigację między poszczególnymi modułami systemu.

Realizacja projektu wiązała się z wieloma wyzwaniami, z których najważniejsze dotyczyły stworzenia spójnej architektury systemu oraz zapewnienia integralności danych przechowywanych w formacie JSON. Przypisywanie unikalnych identyfikatorów oraz wdrożenie mechanizmu tworzenia kopii zapasowych wymagało precyzyjnego podejścia i iteracyjnego testowania. Dodatkowo, walidacja danych wprowadzanych przez użytkowników oraz projektowanie czytelnego interfejsu konsolowego stanowiły istotne aspekty, których celem było zminimalizowanie błędów i zapewnienie stabilności działania systemu.

Podsumowując, projekt umożliwia skuteczne zarządzanie ofertami turystycznymi, rezerwacjami oraz rozliczeniami finansowymi, stanowiąc solidną bazę do dalszych rozwoju. W przyszłości system można udoskonalić poprzez migrację danych do dedykowanej bazy, integrację z zewnętrznymi platformami płatności, wdrożenie graficznego interfejsu użytkownika oraz implementację zaawansowanych mechanizmów bezpieczeństwa, takich jak dwuskładnikowe uwierzytelnianie i szyfrowanie danych. Takie usprawnienia pozwolą na obsługę większych obciążeń oraz podniosą poziom ochrony informacji, co wpłynie na zwiększenie efektywności i atrakcyjności systemu na rynku.


% ********** Koniec rozdziału **********

