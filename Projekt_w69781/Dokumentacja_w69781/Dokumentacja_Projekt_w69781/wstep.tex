\chapter*{Wstęp}

Współczesny rynek turystyczny charakteryzuje się nieustanną dynamiką, a klienci oczekują coraz bardziej zróżnicowanych i dostosowanych do ich potrzeb usług. W związku z tym biura podróży muszą wykorzystywać nowoczesne technologie, które pozwolą im utrzymać konkurencyjność i spełniać wymagania rynku. Wspomaganie codziennej działalności biura podróży za pomocą systemów informatycznych staje się kluczowe w procesach zarządzania ofertami, rezerwacjami, a także w obiegu informacji dotyczących płatności i rozliczeń. Tradycyjne, papierowe metody obsługi mogą prowadzić do błędów, opóźnień i utrudnionego dostępu do danych, co negatywnie wpływa na efektywność pracy oraz satysfakcję klientów. W odpowiedzi na te problemy, niniejszy projekt ma na celu stworzenie systemu informatycznego, który będzie kompleksowo wspierać zarządzanie biurem podróży.

System ten pozwoli na łatwe tworzenie, edytowanie oraz usuwanie ofert turystycznych, umożliwiając szybką reakcję na zmieniające się potrzeby rynku oraz na wprowadzanie nowych produktów do oferty. Dodatkowo, aplikacja będzie umożliwiać dokonywanie rezerwacji biletów na dostępne oferty, co zapewni wygodę klientom oraz pozwoli na pełną kontrolę nad rezerwacjami dla pracowników biura. Ważnym aspektem systemu będzie również moduł umożliwiający obsługę rozliczeń finansowych, który zautomatyzuje procesy związane z płatnościami za usługi oraz umożliwi szybkie i przejrzyste rozliczanie się z klientami.

Projekt zostanie opracowany w języku C\#, co pozwoli na stworzenie wydajnej, bezpiecznej i skalowalnej aplikacji. Wykorzystanie tego języka umożliwia budowę systemu, który będzie łatwy do rozbudowy, dostosowania oraz utrzymania w przyszłości, co stanowi istotny element w kontekście rozwijających się potrzeb biura podróży. Projektowana aplikacja będzie zawierała intuicyjny interfejs użytkownika, który ułatwi pracownikom biura szybkie wykonywanie niezbędnych operacji oraz pozwoli klientom na komfortowe korzystanie z dostępnych usług.

